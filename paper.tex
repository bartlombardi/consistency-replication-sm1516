\documentclass[twoside]{article}

\usepackage{lipsum}

\usepackage[sc]{mathpazo}
\usepackage[utf8]{inputenc}
\linespread{1.05} % Line spacing - Palatino needs more space between lines
\usepackage{microtype} % Slightly tweak font spacing for aesthetics

\usepackage{graphicx}
\usepackage[hmarginratio=1:1,top=32mm,columnsep=20pt]{geometry} % Document margins
\usepackage{multicol} % Used for the two-column layout of the document
\usepackage[hang, small,labelfont=bf,up,textfont=it,up]{caption} % Custom captions under/above floats in tables or figures
\usepackage{booktabs} % Horizontal rules in tables
\usepackage{float} % Required for tables and figures in the multi-column environment - they need to be placed in specific locations with the [H] (e.g. \begin{table}[H])
\usepackage{hyperref} % For hyperlinks in the PDF

\usepackage{lettrine} % The lettrine is the first enlarged letter at the beginning of the text
\usepackage{paralist} % Used for the compactitem environment which makes bullet points with less space between them

\usepackage{abstract} % Allows abstract customization
\renewcommand{\abstractnamefont}{\normalfont\bfseries} % Set the "Abstract" text to bold
\renewcommand{\abstracttextfont}{\normalfont\small\itshape} % Set the abstract itself to small italic text

\usepackage{titlesec} % Allows customization of titles
\renewcommand\thesection{\Roman{section}} % Roman numerals for the sections
\renewcommand\thesubsection{\Roman{subsection}} % Roman numerals for subsections
\titleformat{\section}[block]{\large\scshape\centering}{\thesection.}{1em}{} % Change the look of the section titles
\titleformat{\subsection}[block]{\large}{\thesubsection.}{1em}{} % Change the look of the section titles

\usepackage{fancyhdr} % Headers and footers
\pagestyle{fancy} % All pages have headers and footers
\fancyhead{} % Blank out the default header
\fancyfoot{} % Blank out the default footer
\fancyhead[C]{$\bullet$ Gennaio 2016 $\bullet$} % Custom header text
\fancyfoot[RO,LE]{\thepage} % Custom footer text
\setlength\parindent{0pt}
\setlength\parskip{10pt}

%----------------------------------------------------------------------------------------
%	TITLE SECTION
%----------------------------------------------------------------------------------------

\title{\vspace{-15mm}\fontsize{24pt}{10pt}\selectfont\textbf{Consistency and Replication}} % Article title

\author{
\large
\textsc{Bartolomeo Lombardi, Amerigo Mancino, Andrea Segalini}\\[2mm] % Your name
\normalsize Università degli Studi di Bologna \\ % Your institution
\normalsize \href{mailto:bartolomeo.lombardi@studio.unibo.it}{bartolomeo.lombardi@studio.unibo.it}\\
\normalsize \href{mailto:amerigo.mancino@studio.unibo.it}{amerigo.mancino@studio.unibo.it}\\
\normalsize \href{mailto:andrea.segalini@studio.unibo.it}{andrea.segalini@studio.unibo.it}
\vspace{-5mm}
}
\date{}

%----------------------------------------------------------------------------------------

\begin{document}

\maketitle 
\thispagestyle{fancy} % All pages have headers and footers

%----------------------------------------------------------------------------------------
%	ABSTRACT
%----------------------------------------------------------------------------------------

\begin{abstract}

\noindent
Sistemi di storage moderni replicano i dati su più macchine al fine di garantirne persistenza, low latency e tolleranza ai guasti. La presenza di più repliche di un medesimo dato genera quindi problemi di coerenza fra i dati stessi. Tale coerenza può essere garantita mediante diversi modelli, ognuno con i propri punti di forza e le proprie debolezze. Chiaramente, non ne esiste uno valido in generale e la scelta deve essere fatta tenendo in considerazione quali proprietà del sistema riteniamo maggiormente rilevanti ai nostri scopi. Pertanto, prendendo in esame i due sistemi di key-value store distribuiti Amazon Dynamo e COPS, è stato possibile verificare, anche tramite confronti, come questi compromessi vengono raggiunti.

\end{abstract}

%----------------------------------------------------------------------------------------
%	ARTICLE CONTENTS
%----------------------------------------------------------------------------------------

\begin{multicols}{2} % Two-column layout throughout the main article text

\section{Introduzione}

\lettrine[nindent=0em,lines=3]{O} gni applicazione necessita di un livello di coerenza che cambia a seconda degli scopi che vuole raggiungere e delle priorità che vuole ottenere. Infatti, nel 2000 è stato congetturato, e successivamente dimostrato, un risultato teorico che prova che in ogni sistema distribuito non è possibile soddisfare contemporaneamente tutte e tre le seguenti proprietà: \emph{availability}, ossia ogni richiesta riceve sempre una risposta su ciò che è riuscito o fallito non rimanendo mai in attesa indefinitamente, \emph{partition-tolerance}, ossia il sistema continua a funzionare nonostante arbitrarie perdite di messaggi, \emph{consistency}, ossia tutti i nodi vedono gli stessi dati simultaneamente. Questa conclusione è stata dimostrata nel 2002, dopo quasi trenta anni di ricerca, e va sotto il nome di Teorema CAP.
Per descrivere alcuni differenti modelli di coerenza, utilizzeremo come esempio i punteggi di una partita di baseball, memorizzati all'interno di un key-value store distribuito, come mostrato di seguito:\\
\begin{table*}[htb]
\resizebox{\columnwidth}{!}{%
\begin{tabular}{l*{10}{c}r}
 \textbf{Team} & \textbf{1} & \textbf{2} & \textbf{3} & \textbf{4} & \textbf{5} & \textbf{6} & \textbf{7} & \textbf{8} & \textbf{9} & \textbf{RUNS} \\
 \hline
  \textbf{Visitors}     & 0 & 0 & 1 & 0 & 1 & 0 & 0 &  &  & 2 \\
  \textbf{Home}         & 1 & 0 & 1 & 1 & 0 & 2 &   &  &  & 5 \\
\end{tabular}
}
\end{table*}

\textbf{Strong consistency.} La garanzia più alta che possiamo raggiungere fornisce ad ogni client che effettua operazioni di lettura sempre l'ultimo valore aggiornato. Per implementare tale livello, è necessario un alto livello di sincronizzazione tra i vari nodi, che per essere raggiunto esige attese, causando un calo delle performance e la partecipazione attiva di tutti i nodi. In riferimento all'esempio, è possibile che ci venga ritornato unicamente il punteggio 2-5.

\textbf{Eventual Consistency.} Formalmente l'eventual consistency consente di ritornare un qualunque valore che è stato scritto su un dato che il client vuole leggere. In pratica, quello che questo livello garantisce è che se non vengono eseguiti nuovi aggiornamenti su un oggetto, eventualmente tutti gli accessi a quell'oggetto ritorneranno l'ultimo valore aggiornato.


%----------------------------------------------------------------------------------------
%	REFERENCE LIST
%----------------------------------------------------------------------------------------

\begin{thebibliography}{99} % Bibliography - this is intentionally simple in this template

\bibitem[Figueredo and Wolf, 2009]{Figueredo:2009dg}
Figueredo, A.~J. and Wolf, P. S.~A. (2009).
\newblock Assortative pairing and life history strategy - a cross-cultural
  study.
\newblock {\em Human Nature}, 20:317--330.
 
\end{thebibliography}

%----------------------------------------------------------------------------------------

\end{multicols}

\end{document}
