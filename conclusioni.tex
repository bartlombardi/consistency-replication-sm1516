\begin{frame}
\frametitle{Conclusioni}
In un sistema distribuito ci sono sempre due prospettive da tenere in considerazione:
\begin{itemize}
	\item Il provider \\
		  Vede lo stato interno del sistema: la sua priorità è la sincronizzazione
		  dei processi fra le repliche e l'ordine delle operazioni;
	\item Un client del sistema di storage \\
		  Il sistema è una black-box: la sua attenzione è rivolta alle garanzie che
		  il sistema distribuito deve fornirgli come parte di un SLA.
\end{itemize}
Quindi, a seconda delle necessità dei diversi sistemi e della prospettiva client-centrica
o data-centrica che scegliamo di adottare, possiamo implementare diversi livelli di coerenza.
\end{frame}