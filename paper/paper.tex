% pdflatex paper.tex
% biber paper.bcf
% pdflate paper.tex (x2)

\documentclass[twoside]{article}

\usepackage[sc]{mathpazo}
\usepackage[utf8]{inputenc}
\usepackage[english,italian]{babel}
\linespread{1.05} % Line spacing - Palatino needs more space between lines
\usepackage{microtype} % Slightly tweak font spacing for aesthetics

\usepackage{graphicx}
\usepackage[hmarginratio=1:1,top=32mm,columnsep=20pt]{geometry} % Document margins
\usepackage{multicol} % Used for the two-column layout of the document
\usepackage[hang, small,labelfont=bf,up,textfont=it,up]{caption} % Custom captions under/above floats in tables or figures
\usepackage{booktabs} % Horizontal rules in tables
\usepackage{float} % Required for tables and figures in the multi-column environment - they need to be placed in specific locations with the [H] (e.g. \begin{table}[H])
\usepackage{hyperref} % For hyperlinks in the PDF

\usepackage{lettrine} % The lettrine is the first enlarged letter at the beginning of the text
\usepackage{paralist} % Used for the compactitem environment which makes bullet points with less space between them

\usepackage{abstract} % Allows abstract customization
\renewcommand{\abstractnamefont}{\normalfont\bfseries} % Set the "Abstract" text to bold
\renewcommand{\abstracttextfont}{\normalfont\small\itshape} % Set the abstract itself to small italic text

\usepackage{titlesec} % Allows customization of titles
\renewcommand\thesection{\Roman{section}} % Roman numerals for the sections
\renewcommand\thesubsection{\Roman{subsection}} % Roman numerals for subsections
\titleformat{\section}[block]{\large\scshape\centering}{\thesection.}{1em}{} % Change the look of the section titles
\titleformat{\subsection}[block]{\large}{\thesubsection.}{1em}{} % Change the look of the section titles
%\usepackage{fancyhdr} % Headers and footers
%\pagestyle{fancy} % All pages have headers and footers
%\fancyhead{} % Blank out the default header
%\fancyfoot{} % Blank out the default footer
%\fancyhead[C]{$\bullet$ Gennaio 2016 $\bullet$} % Custom header text
%\fancyfoot[RO,LE]{\thepage} % Custom footer text

\usepackage{amsmath}
\usepackage{txfonts}
\usepackage{enumitem}

\usepackage{subcaption}

\usepackage{biblatex}
\bibliography{bibliography}

\setlength\parindent{0pt} % Nessuna rientranza ad inizio paragrafo
\setlength\parskip{10pt} % Spaziatura tra paragrafi

%----------------------------------------------------------------------------------------
% IMPOSTAZIONI DI SILLABAZIONE
%----------------------------------------------------------------------------------------

\hyphenation{dynamo baseball cluster consistency COPS replication availability partition-tolerance Communications SIGOPS annual}
% Inserire le parole che non devono essere sillabate e spezzate da tex

%----------------------------------------------------------------------------------------
%	CUSTOM MACRO
%----------------------------------------------------------------------------------------

\newcommand{\vclock}{\mathrm{VC}} % VC

%----------------------------------------------------------------------------------------
%	TITLE SECTION
%----------------------------------------------------------------------------------------

\title{\vspace{-15mm}\fontsize{24pt}{10pt}\selectfont\textbf{Consistency and Replication}} % Article title

\author{
\large
\textsc{Bartolomeo Lombardi, Amerigo Mancino, Andrea Segalini}\\[2mm] % Your name
\normalsize Università degli Studi di Bologna % Your institution
\vspace{5mm} \\
\normalsize \href{mailto:bartolomeo.lombardi@studio.unibo.it}{bartolomeo.lombardi@studio.unibo.it}\\
\normalsize \href{mailto:amerigo.mancino@studio.unibo.it}{amerigo.mancino@studio.unibo.it}\\
\normalsize \href{mailto:andrea.segalini@studio.unibo.it}{andrea.segalini@studio.unibo.it}
\vspace{-5mm}
}
\date{Gennaio 2016}

%----------------------------------------------------------------------------------------

\begin{document}
\maketitle 
%\thispagestyle{fancy} % All pages have headers and footers

%----------------------------------------------------------------------------------------
%	ABSTRACT
%----------------------------------------------------------------------------------------

\begin{abstract}
\noindent
Sistemi di storage moderni replicano i dati su più macchine al fine di garantirne persistenza, low latency e tolleranza ai guasti. La presenza di più repliche di un medesimo dato genera quindi problemi di coerenza fra i dati stessi. Tale coerenza può essere garantita mediante diversi modelli, ognuno con i propri punti di forza e le proprie debolezze. Chiaramente, non ne esiste uno valido in generale e la scelta deve essere fatta tenendo in considerazione quali proprietà del sistema riteniamo maggiormente rilevanti ai nostri scopi. Pertanto, prendendo in esame i due sistemi di key-value store distribuiti Amazon Dynamo e COPS, è stato possibile verificare, anche tramite confronti, come questi compromessi vengono raggiunti.
\end{abstract}

%----------------------------------------------------------------------------------------

\vspace{20mm} % Spaziatura tra abstract e corpo articolo

%----------------------------------------------------------------------------------------
%	ARTICLE CONTENTS
%----------------------------------------------------------------------------------------

\begin{multicols}{2} % Two-column layout throughout the main article text

\section{Introduzione}
\label{sec:introduzione}
\lettrine[nindent=0em,lines=2]{O} gni applicazione necessita di un livello di coerenza che cambia a seconda degli scopi che vuole raggiungere e delle priorità che vuole ottenere. Infatti, nel 2000 è stato congetturato, e successivamente dimostrato, un risultato teorico che prova che in ogni sistema distribuito non è possibile soddisfare contemporaneamente tutte e tre le seguenti proprietà: \emph{availability}, ossia ogni richiesta riceve sempre una risposta su ciò che è riuscito o fallito non rimanendo mai in attesa indefinitamente, \emph{partition-tolerance}, ossia il sistema continua a funzionare nonostante arbitrarie perdite di messaggi, \emph{consistency}, ossia tutti i nodi vedono gli stessi dati simultaneamente. Questa conclusione è stata dimostrata nel 2002, dopo quasi trenta anni di ricerca, e va sotto il nome di Teorema CAP.
Per descrivere alcuni differenti modelli di coerenza, utilizzeremo come esempio i punteggi di una partita di baseball, memorizzati all'interno di un key-value store distribuito, come mostrato di seguito:
\begin{table}[H]
\centering
\resizebox{\columnwidth}{!}{%
\begin{tabular}{l*{10}{c}r}
\textbf{Team} & \textbf{1} & \textbf{2} & \textbf{3} & \textbf{4} & \textbf{5} & \textbf{6} & \textbf{7} & \textbf{8} & \textbf{9} & \textbf{RUNS} \\
\hline
\textbf{Visitors}     & 0 & 0 & 1 & 0 & 1 & 0 & 0 &  &  & 2 \\
\textbf{Home}         & 1 & 0 & 1 & 1 & 0 & 2 &   &  &  & 5 \\
\end{tabular}
}
\caption{Esempio di punteggio nel baseball.}
\label{tab:punteggio-baseball}
\end{table}

\textbf{Strong consistency.} La garanzia più alta che possiamo raggiungere fornisce ad ogni client che effettua operazioni di lettura sempre l'ultimo valore aggiornato. Per implementare tale livello, è necessario un alto livello di sincronizzazione tra i vari nodi, che per essere ottenuto esige attese, causando un calo delle performance e la partecipazione attiva di tutti i nodi. In riferimento all'esempio, è possibile che ci venga ritornato unicamente il punteggio $2$-$5$.

\textbf{Eventual Consistency.} Formalmente l'eventual consistency consente di ritornare un qualunque valore che è stato scritto su un dato che il client vuole leggere. In pratica, quello che questo livello garantisce è che se non vengono eseguiti nuovi aggiornamenti su un oggetto, eventualmente tutti gli accessi a quell'oggetto ritorneranno l'ultimo valore aggiornato.

\textbf{Consistent Prefix.} Richiedendo consistent prefix, è garantito osservare una sequenza ordinata di scritture nel sistema. Eseguendo un'operazione di lettura, quindi, si osserva una versione del data store che è esistita in un qualche momento nel passato e i valori ritornati saranno coerenti con quella versione del data store. Nel nostro esempio, potranno esserci ritornati allora i punteggi $0$-$0$, $0$-$1$, $1$-$1$, $1$-$2$, $1$-$3$, $2$-$3$, $2$-$4$, $2$-$5$.

\textbf{Bounded Staleness.} Definito un lasso di tempo $T$ nel sistema, la bounded staleness garantisce che un'operazione di lettura non ritorni un valore scritto prima di $T$. Quindi, prendendo come $T$ un inning, nel nostro esempio ci possono essere ritornati i punteggi $2$-$3$, $2$-$4$, $2$-$5$.

\textbf{Monotonic Reads.} Se un client esegue un'operazione di lettura su un oggetto, è garantito che una successiva lettura sul medesimo oggetto restituirà lo stesso valore letto in precedenza o una sua versione successiva, mai precedente. In riferimento sempre al nostro esempio, dopo aver letto allora il punteggio $1$-$3$, potrà venir restituito ancora il punteggio $1$-$3$ oppure un punteggio successivo.

\textbf{Read My Writes.}	Garantisce strong consistency negli aggiornamenti che un client scrive, eventual consistency negli altri casi. In altre parole, chi scrive leggerà il punteggio $2$-$5$, mentre per gli altri varrà quanto già detto per l'eventual consistency.

\textbf{Causal Consistency.} Operazioni di memoria che potenzialmente sono causalmente collegate vengono viste da ogni nodo del sistema nello stesso ordine. Praticamente, definiamo una relazione fra operazioni, indicata con il simbolo $\rightsquigarrow$, che gode delle seguenti proprietà:
\begin{enumerate}[topsep=0pt,itemsep=-1ex,partopsep=1ex,parsep=1ex]
\item Execution thread. Se a e b sono due operazioni in un singolo thread di esecuzione, allora a $\rightsquigarrow$ b se e sono se a accade prima di b.
\item Gets from. Se a è un'operazione di \texttt{put} e b è un'operazione di \texttt{get} che ritorna il valore scritto da a allora vale a $\rightsquigarrow$ b.
\item Transitivity. Se a, b e c sono tre operazioni e a $\rightsquigarrow$ b e b $\rightsquigarrow$ c, allora anche a $\rightsquigarrow$ c.
\end{enumerate}
I valori restituiti da una replica, in sostanza, sono coerenti con l'ordine appena definito da $\rightsquigarrow$.

\section{Dettagli Tecnici}
\label{sec:dettagli-tecnici}
%\lettrine[nindent=0em,lines=2]{I}
In questa sezione prendiamo in considerazione due noti sistemi di storage distribuiti: \emph{COPS} (\emph{Cluster of Order Preserving Service})\cite{bib:COPS} e \emph{Amazon Dynamo}\cite{bib:dynamo}. Il primo è un sistema distribuito su scala geografica (quella che viene definita in gergo come “wide area”) con repliche identiche degli stessi dati dislocate in più datacenter. Il secondo, invece, è costituito da nodi, potenzialmente disposti in più datacenter distinti, in cui sono distribuiti e replicati i dati. La specifica afferma che in COPS questi dati non sono partizionati fra i vari datacenter che formano il sistema, ma replicati tra essi, mentre in Dynamo sono divisi nei diversi nodi e di ogni elemento ne esistono più copie.

\textbf{Replicazione.} Per affrontare il problema della replicazione dobbiamo innanzi tutto sottolineare come in COPS ogni datacenter possieda una replica di tutti i dati, mentre in Dynamo la questione diventa più complessa, in quanto i dati non sono replicati su tutti i nodi ma solo su un numero $N$ (dove $N$ è un intero molto minore del numero totale di nodi di cui il sistema dispone). Inoltre, in quest'ultimo caso il partizionamento dei dati viene gestito tramite la tecnica del \emph{Consistent Hashing} \cite{bib:hashing}, che considera una tabella hash chiusa ad anello e con nodi disposti su di esso, al fine di delimitare gli intervalli con cui sono partizionati i dati (figura \ref{fig:consistent-hashing-ring}).
\begin{figure}[H]
\centering
\resizebox{\columnwidth}{!}{%
\includegraphics{img/consistent-hashing-ring.png}
}
\caption{Schematizzazione del Consistent Hashing}
\label{fig:consistent-hashing-ring}
\end{figure}

\textbf{Versionamento.} La presenza di repliche necessita l'aggiunta di informazioni ausiliarie per tracciare le versione e poter risolvere eventuali conflitti. In COPS ad ogni aggiornamento viene associato un \emph{Lamport timestamp} \cite{bib:lamport}, costituito da un clock logico e dall'identificativo univoco del nodo, che è tale da fornire un ordinamento globale. Mediante questo espediente, è possibile implementare una \emph{last-writer-win-rule}, garantendo, in caso di conflitti, la vittoria della scrittura con timestamp maggiore. In Dynamo invece la tecnica adottata è quella dei \emph{Vector clock} \cite{bib:lamport} che introduce un vettore di coppie $\langle\mathrm{id\char`_nodo}, \mathrm{counter}\rangle$, abbinato ad ogni dato replicato. Ogni volta che un nodo compie una modifica, incrementa il proprio counter; date quindi due versioni $x$ e $y$ del medesimo oggetto e denotato con $\vclock(x)_z$ il vector clock relativo al nodo $z$, abbiamo che vale:
\begin{multline*}
\vclock(x) < \vclock(y) \iff \\
\big( \forall z \mathrel{:} \vclock(x)_z \leq \vclock(y)_z \big) \\
\land \big( \exists z' \mathrel{:} \vclock(x)_{z'} < \vclock(y)_{z'} \big)
\end{multline*}
Chiaramente, l'approccio intrapreso da Dynamo consente di tracciare branch di versioni parallele ed operare una riconciliazione definita dal client, mentre in COPS, data la natura totale dell'ordinamento, in caso di conflitto, verrebbe sicuramente perso uno dei due rami divergenti. In quest'ultimo caso, la scelta fatta garantisce quindi una notevole semplicità nell'applicazione, mentre, al contrario, in Dynamo è necessario prevedere e gestire la ricezione di versioni concorrenti. Inoltre, la dimensione del vector clock può crescere molto, soprattutto in presenza di numerosi fallimenti, rendendo onerose le operazioni di analisi su di essi.

\textbf{Coerenza.} In COPS implementiamo \emph{causal+ consistency}, in quanto si vuole garantire la coerenza più forte possibile che permetta, al contempo, di soddisfare le caratteristiche \emph{ALPS}, nel rispetto del Teorema CAP. È impossibile quindi implementare livelli di coerenza maggiori, quali la strong consistency o la sequential consistency. In dettaglio, la causal+ consistency viene definita come la combinazione di due componenti: la causal consistency, definita nella Sezione \ref{sec:introduzione}, e la \emph{convergent conflict handling}, che garantisce che tutte le operazioni di \texttt{put} in conflitto vengano gestite allo stesso modo da tutti i nodi usando una funzione $h$ associativa e commutativa.
In Dynamo, al contrario, sono implementati due modelli di coerenza, strong consistency ed eventual consistency, impostabili mediante dei parametri che governano direttamente anche performance, availability e persistenza dei dati. Questi ultimi parametri, sempre nel rispetto del Teorema CAP, consentono di bilanciare il sistema, permettendone un diverso uso in base alle esigenze.

\textbf{Coerenza in Dynamo}. La coerenza delle operazioni è gestita con la tecnica del \emph{Quorum}\cite{bib:quorum}. Il sistema permette di impostare due parametri: il numero di nodi dai quali viene atteso un acknowledge a seguito di una scrittura, detto $W$, e il numero di nodi che partecipa ad un'operazione di lettura, detto $R$. Facendo riferimento alla Figura \ref{fig:get}, nell'operazione di lettura, il client contatta il nodo A che diventa il coordinatore della richiesta e poi i restanti $N-1$ nodi che mantengono la replica del dato. Dopo aver cercato nella propria memoria secondaria il valore, A aspetta la risposta da almeno $R-1$ nodi (non $R$, poiché A stesso è conteggiato). I valori ritornati, a questo punto, possono appartenere tutti alla medesima versione, quindi essere tutti coerenti, oppure presentare versioni precedenti/successive o concorrenti. L'ordinamento parziale dalle versioni è ottenuto analizzando i vector clock degli oggetti. In caso di ordinamento esistente, il conflitto è risolto da A prendendo l'ultima versione, viceversa sono tornate al client tutte le versioni concorrenti collezionate, lasciando a quest'ultimo il compito di riunificare la biforcazione. L'operazione di scrittura (Figura \ref{fig:put}) avviene in maniera analoga: a differenza dei valori vengono aspettati gli acknowledge di conferma sulla scrittura avvenuta.
%% VERSIONE NON AFFIANCATE
%% \begin{figure}[H]
%% \centering
%% \resizebox{\columnwidth}{!}{%
%% \includegraphics{img/get.png}
%% }
%% \caption{Timeline dell'esecuzione di \texttt{get(k)}}
%% \label{fig:get}
%% \end{figure}
%% \begin{figure}[H]
%% \centering
%% \resizebox{\columnwidth}{!}{%
%% \includegraphics{img/put.png}
%% }
%% \caption{Timeline dell'esecuzione di \texttt{put(k, v)}}
%% \label{fig:put}
%% \end{figure}
%%
%% VERSIONE CON BORDI AFFIANCATE
\begin{figure*}[t]
\centering
\begin{minipage}[b]{0.48\textwidth}
\fbox{
\resizebox{0.95\textwidth}{!}{%
\includegraphics{img/get.png}
}
}
\caption{Timeline dell'esecuzione di \texttt{get(k)}}
\label{fig:get}
\end{minipage}
\enskip
\begin{minipage}[b]{0.48\textwidth}
\fbox{
\resizebox{0.95\textwidth}{!}{%
\includegraphics{img/put.png}
}
}
\caption{Timeline dell'esecuzione di \texttt{put(k, v)}}
\label{fig:put}
\end{minipage}
\end{figure*}
%%
%% VERSIONE SENZA BORDI AFFIANCATE
%% \begin{figure*}[t]
%% \centering
%% \begin{minipage}[b]{0.45\textwidth}
%% \resizebox{\columnwidth}{!}{%
%% \includegraphics{img/get.png}
%% }
%% \caption{Timeline dell'esecuzione di \texttt{get(k)}}
%% \label{fig:get}
%% \end{minipage}
%% \enskip
%% \begin{minipage}[b]{0.45\textwidth}
%% \resizebox{\columnwidth}{!}{%
%% \includegraphics{img/put.png}
%% }
%% \caption{Timeline dell'esecuzione di \texttt{put(k, v)}}
%% \label{fig:put}
%% \end{minipage}
%% \end{figure*}
I parametri $W$ e $R$ mediano tra coerenza, availability, performance e persistenza dei dati. Se $W+R > N$ allora l'insieme dei nodi che hanno effettuato sicuramente l'ultima scrittura nel sistema si interseca con l'insieme dei nodi che rispondono in caso di lettura. Dunque abbiamo garanzia che l'ultimo valore scritto per una certa chiave sia ritornato sicuramente in caso di lettura poiché esiste almeno un nodo che lo possiede e che risponde alla richiesta. La garanzia di ritorno dell'ultima versione scritta corrisponde ad una strong consistency \cite{bib:vogels-blog}. Come ripetuto diverse volte, avere tale modello di coerenza implica la perdita di altre qualità. Alti valori di $W$ e $R$ comportano performance peggiori poiché sono attese un maggior numero di risposte ad ogni richiesta, allo stesso modo peggiora l'availability poiché sono necessari un maggior numero di nodi per completare un'operazione. In generale il sistema è usato in configurazione $W+R \leq N$ che implica il modello eventual consistency. Tuttavia più il valore $W+R$ si avvicina ad $N$, più aumenta la probabilità che i due insiemi sopracitati si intersechino e quindi l'ultimo valore letto sia anche l'ultimo scritto.

\textbf{Coerenza in COPS}.
Ogni client possiede un contesto in cui vengono tracciate tutte le dipendenze causali fra le operazioni man mano che sono eseguite azioni di \texttt{put} e di \texttt{get}, creando quindi una rete come quella mostrata in Figura \ref{fig:dipendenze}: se si volesse effettuare un'operazione di \texttt{put} su $z_4$ dovrebbero essere controllate e soddisfatte tutte le sue dipendenze causali prima di procedere. Per minimizzare l'overhead che scaturisce da questo controllo, COPS si affida alle sole ``nearest dependencies'' (letteralmente ``dipendenze più vicine''), per cui, nel caso di $z_4$ diventa sufficiente verificare unicamente $y_1$ e $v_6$, dal momento che, ad esempio, se $v_6$ risulta soddisfatta allora lo risultano anche $t_2$ e $u_1$ per causal consistency.
In COPS, come possiamo vedere in Figura \ref{fig:COPSdesign}, ogni datacenter possiede un cluster locale a cui i client possono far riferimento mediante la library associata.
Quando viene eseguita un'operazione di \texttt{put} al cluster locale, vengono controllate le dipendenze causali mediante un'operazione di \texttt{dep\_check} e, quindi, la scrittura ha effetto. Successivamente, in maniera asincrona, le richieste vengono inserite in una replication queue e spedite agli altri datacenter nella wide area, i quali, una volta ricevute, eseguiranno i dovuti controlli delle dipendenze causali (sempre mediante \texttt{dep\_check}), risolveranno eventuali conflitti e renderanno effettive le modifiche.
Citiamo, a questo proposito, una variante del sistema, denominata COPS-GT, che fornisce anche un'operazione di \texttt{get\_trans}, la quale permette di ritornare una visione coerente di un insieme di chiavi.
\begin{figure}[H]
\centering
\resizebox{\columnwidth}{!}{%
\includegraphics{img/dipendenze.png}
}
\caption{Un semplice grafo delle dipendenze}
\label{fig:dipendenze}
\end{figure}
Queste transazioni sono non bloccanti, non richiedono lock e vengono realizzate al più in due round. 
Nel primo vengono ritornate le ultime versioni disponibili nel cluster locale (insieme alle loro dipendenze), poi, dal momento che possono essere presenti ancora inconsistenze fra due elementi di questo gruppo, nel secondo round sono recuperate le versioni più recenti rispetto a quelle ritornate nel primo.
\begin{figure*}[t]
\centering
\includegraphics[width=\textwidth,height=4cm]{img/COPSdesign.png}
\caption{Design del sistema COPS}
\label{fig:COPSdesign}
\end{figure*}

\section{Conclusioni}
In conclusione, COPS è utilizzato quando è necessario replicare i dati al fine di avvicinarli a più basi di client, ottenendo performance migliori sulle operazioni che sono eseguite localmente. L'unico modello di coerenza implementato, quello della causal+, fornisce un buon compromesso, essendo secondo solo alla strong e alla sequential consistency e garantendo al contempo rapide operazioni attraverso un oculato sistema di aggiornamento asincrono delle repliche (descritto in Sezione \ref{sec:dettagli-tecnici}). La persistenza dei dati è elevata, tollerando al meglio il fallimento di interi datacenter e contemplando la perdita dei soli aggiornamenti in coda, in attesa di essere spediti.
Dynamo, al contrario, si presenta come un ambiente meglio adattabile alle situazioni: non riesce ad implementare lo stesso livello di coerenza di COPS in quanto non supporta la nozione di ``dipendenza'' ma è in grado di garantire strong consistency (a scapito di una pesante degradazione di performance e availability). Inoltre può essere configurato topologicamente per assomigliare a COPS, replicando i dati su tutti i nodi (impostando cioè N come il numero di nodi) e localizzandoli su scala geografica. Mantenere un modello di coerenza forte è molto oneroso per ampie distanze: infatti l'eventual, a differenza della causal+, che assicura coerenza dove è strettamente necessaria, mantiene ridotte le attese di sincronizzazione ma risulta essere molto fragile a fronte del lento completamento delle operazioni di aggiornamento.
Dynamo si rivela quindi un sistema altamente flessibile e configurabile e, dall'analisi condotta, si è capito che simulando COPS tramite esso pagheremmo un prezzo troppo oneroso al fine di garantire una coerenza maggiore.

In generale, in un sistema distribuito ci sono sempre due punti di vista da tener presente: il provider, interessato alla sincronizzazione dei processi fra le repliche e all'ordinamento delle operazioni, e l'ottica di un client, la cui priorità è rivolta alle garanzie che il sistema deve fornirgli come parte magari di un SLA. Quindi, a seconda delle necessità, degli interessi e della prospettiva client-centrica o data-centrica che scegliamo di adottare, verrà selezionato il modello di coerenza più congeniale.

\label{sec:conclusioni}
%----------------------------------------------------------------------------------------
%	REFERENCE LIST
%----------------------------------------------------------------------------------------

%% \begin{thebibliography}{99}

%% \bibitem[Figueredo and Wolf, 2009]{bib:consistent-hashing}
%% Figueredo, A.~J. and Wolf, P. S.~A. (2009).
%% \newblock Assortative pairing and life history strategy - a cross-cultural
%%   study.
%% \newblock {\em Human Nature}, 20:317--330.
 
%% \end{thebibliography}

\printbibliography

%----------------------------------------------------------------------------------------

\end{multicols}

\end{document}
