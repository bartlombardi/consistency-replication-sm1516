% pdflatex paper.tex
% biber paper.bcf
% pdflate paper.tex (x2)

\documentclass[twoside]{article}

\usepackage[sc]{mathpazo}
\usepackage[utf8]{inputenc}
\usepackage[english,italian]{babel}
\linespread{1.05} % Line spacing - Palatino needs more space between lines
\usepackage{microtype} % Slightly tweak font spacing for aesthetics

\usepackage{graphicx}
\usepackage[hmarginratio=1:1,top=32mm,columnsep=20pt, margin=2.5cm]{geometry} % Document margins
%\usepackage{multicol} % Used for the two-column layout of the document
\usepackage[hang, small,labelfont=bf,up,textfont=it,up]{caption} % Custom captions under/above floats in tables or figures
\usepackage{booktabs} % Horizontal rules in tables
\usepackage{float} % Required for tables and figures in the multi-column environment - they need to be placed in specific locations with the [H] (e.g. \begin{table}[H])
\usepackage{hyperref} % For hyperlinks in the PDF

\usepackage{lettrine} % The lettrine is the first enlarged letter at the beginning of the text
\usepackage{paralist} % Used for the compactitem environment which makes bullet points with less space between them

\usepackage{abstract} % Allows abstract customization
\renewcommand{\abstractnamefont}{\normalfont\bfseries} % Set the "Abstract" text to bold
\renewcommand{\abstracttextfont}{\normalfont\small\itshape} % Set the abstract itself to small italic text

\usepackage{titlesec} % Allows customization of titles
\renewcommand\thesection{\Roman{section}} % Roman numerals for the sections
\renewcommand\thesubsection{\Roman{subsection}} % Roman numerals for subsections
\titleformat{\section}[block]{\large\scshape\centering}{\thesection.}{1em}{} % Change the look of the section titles
\titleformat{\subsection}[block]{\large}{\thesubsection.}{1em}{} % Change the look of the section titles
%\usepackage{fancyhdr} % Headers and footers
%\pagestyle{fancy} % All pages have headers and footers
%\fancyhead{} % Blank out the default header
%\fancyfoot{} % Blank out the default footer
%\fancyhead[C]{$\bullet$ Gennaio 2016 $\bullet$} % Custom header text
%\fancyfoot[RO,LE]{\thepage} % Custom footer text

\usepackage{amsmath}
\usepackage{txfonts}
\usepackage{enumitem}

\usepackage{subcaption}

\usepackage{biblatex}
\bibliography{bibliography}

\setlength\parindent{0pt} % Nessuna rientranza ad inizio paragrafo
\setlength\parskip{10pt} % Spaziatura tra paragrafi

%----------------------------------------------------------------------------------------
% IMPOSTAZIONI DI SILLABAZIONE
%----------------------------------------------------------------------------------------

\hyphenation{dynamo baseball cluster consistency COPS replication availability partition-tolerance Communications SIGOPS annual preserving performance}
% Inserire le parole che non devono essere sillabate e spezzate da tex

%----------------------------------------------------------------------------------------
%	CUSTOM MACRO
%----------------------------------------------------------------------------------------

\newcommand{\vclock}{\mathrm{VC}} % VC

%----------------------------------------------------------------------------------------
%	TITLE SECTION
%----------------------------------------------------------------------------------------

\title{\vspace{-15mm}\fontsize{24pt}{10pt}\selectfont\textbf{Configurazione di application server in ambiente di cloud computing}} % Article title

\author{
\large
\textsc{Bartolomeo Lombardi, Amerigo Mancino, Andrea Segalini}\\[2mm] % Your name
\normalsize Università degli Studi di Bologna % Your institution
\vspace{5mm} \\
\normalsize \href{mailto:bartolomeo.lombardi@studio.unibo.it}{bartolomeo.lombardi@studio.unibo.it}\\
\normalsize \href{mailto:amerigo.mancino@studio.unibo.it}{amerigo.mancino@studio.unibo.it}\\
\normalsize \href{mailto:andrea.segalini@studio.unibo.it}{andrea.segalini@studio.unibo.it}
\vspace{-5mm}
}
\date{Anno Accademico 2015-16}

%----------------------------------------------------------------------------------------

\begin{document}
\maketitle 
%\thispagestyle{fancy} % All pages have headers and footers

%----------------------------------------------------------------------------------------
%	ABSTRACT
%----------------------------------------------------------------------------------------

\begin{abstract}
\noindent
OpenNebula è un toolkit che offre una soluzione semplice ma flessibile per la costruzione e la gestione
di soluzioni cloud e di centri di elaborazione dati distribuiti eterogenei.
In questa breve relazione ci proponiamo di illustrare, passo per passo, il lavoro fatto, consistente
nell'installazione e nella configurazione dei software OpenNebula e JBoss, oltre che nella scrittura
di un piccolo programma JBoss volto a dimostrare le proprietà di tolleranza ai guasti e migrazione.
Verranno poi discusse le varie problematiche incontrate e saranno mostrati i metodi che hanno portato
alla loro risoluzione.
\end{abstract}

%----------------------------------------------------------------------------------------

\vspace{15mm} % Spaziatura tra abstract e corpo articolo %% backup {20mm}

%----------------------------------------------------------------------------------------
%	ARTICLE CONTENTS
%----------------------------------------------------------------------------------------

\section{Introduzione: tecnologie utilizzate}
Per creare un sistema di cloud e dimostrare quindi le proprietà di migrazione e
tolleranza ai guasti volute ci si è serviti di due tecnologie principali:
\begin{itemize}
	\item OpenNebula \\
		  Progetto open source per sistemi GNU/Linux supportato da una vasta community, in grado di
		  creare e gestire strutture cloud e data center virtualizzati. Differentemente da altre soluzioni,
		  OpenNebula si pone come un sistema aperto, flessibile e personalizzabile in base alle esigenze
		  oltre che robusto, in grado di fornire anche funzionalità innovative per cloud privati e ibridi.
		  Inoltre permette la realizzazione di cloud Iaas (Infrastructure as a Service), permettendo integrità
		  anche nel caso di macchine eterogenee dal punto di vista hardware e software. Come unico requisito
		  richiede un hardware con supporto per la virtualizzazione.
	\item JBoss \\
		  WildFly, precedentemente noto come JBoss, è un application server open source che implementa
		  le specifiche Java EE. Fra le numerose componenti di cui è costituito il prodotto, sottolineiamo
		  le funzionalità di clustering (che funge da load balancer), di failover (invio di componenti di
		  un sistema ad una seconda componente quando la prima presenta qualche problema), di distributed
		  caching, oltre che di implementazione distribuita. Tutte queste caratteristiche la rendono una
		  ottima piattaforma di sviluppo.
\end{itemize}

\end{document}
