\section{Replicated Data Consistency Explained Through Baseball}

\begin{frame}
 \frametitle{Replica di dati nel cloud}
 \begin{itemize}
   \item Sistemi di storage replicano i dati su più macchine per la resistenza ai guasti e per migliorare le performance.
   \item Si utilizza la geo-replication per resistere ad una eventuale perdita di datacenter interi.
   \item I datacenter vengono distribuiti a livello globale in più continenti per garantire che i dati siano vicini a più basi di client.
 \end{itemize}
  
\end{frame}

\begin{frame}
 \frametitle{Sistemi di storage popolari}
Di seguito presentiamo alcuni esempi di sistemi e il loro grado di coerenza implementato:
 \begin{itemize}
   \item Amazon S3 – eventual consistency
   \item Amazon Simple DB – eventual o strong
   \item Google App Engine – strong o eventual
   \item Yahoo! PNUTS – eventual o strong
   \item Windows Azure Storage – strong o eventual
 \end{itemize}
In questa presentazione:
\begin{itemize}
  \item Amazon Dynamo - strong o eventual
  \item COPS - casual+ consistency
\end{itemize}
%Amazon ha spinto sulla eventual consistency, ma offre la strong consistency come opzione per servizi, come SimpleDB e DynamoDB.
%Google App Engine ha cominciato offrendo solo strong consistency, ma dopo ha aggiunto delle opzioni di lettura, weakly consistent data per quelle applicazioni che richiedono maggiori performance.
%Yahoo!’s PNUTS system, la tencologia di base per molti dei servizi offerti da Yahoo, offre sia eventual consistency e strong consistency.
%Windows Azure fornisce solo la strong consistency, anche se utilizza l'eventual consistency per i dati geo-replicati.
\end{frame}

\begin{frame}
\frametitle{Teorema CAP (Teorema di Bremer)}
	\begin{definizione}
	 Il \alert{teorema CAP} afferma che è impossibile per un sistema informatico distribuito fornire simultaneamente tutte e tre le seguenti garanzie:
	\begin{itemize}
		\item \textbf{Consistency}\\
		Tutti i nodi vedono gli stessi dati nello stesso momento
		\item \textbf{Availability}\\
	    La garanzia che ogni richiesta riceva una risposta su ciò che sia riuscito o fallito
		\item \textbf{Partition Tollerance}\\
		Il sistema continua a funzionare nonostante arbitrarie perdite di messaggi
	\end{itemize}
	Secondo il teorema, un sistema distribuito è in grado di soddisfare al massimo due di queste garanzie allo stesso tempo, ma mai tutte e tre.
	\end{definizione}
\end{frame}

\begin{frame}
\frametitle{Alcuni tipi di coerenza I}
\begin{itemize}
  \item \textbf{Strong Consistency}: garantisce che l'operazione di lettura ritorni il valore dell'ultima scrittura.
  \item \textbf{Eventual Consistency}: è la più debole delle altre, poiché permette di avere come valore di ritorno il più grande insieme di possibilità.
\end{itemize}
\end{frame}

\begin{frame}
\frametitle{Alcuni tipi di coerenza II}
\begin{itemize}
  \item \textbf{Consistent Prefix}: data una sequenza di modifiche ritorna l'ultima a lui nota.
  \item \textbf{Bounded Staleness}: assicura di ritornare il valore che non sia più vecchio di un lasso di tempo $T$
\end{itemize}
\end{frame}

\begin{frame}
\frametitle{Alcuni tipi di coerenza III}
\begin{itemize}
  \item \textbf{Monotonic Reads}: se il client effettua la lettura di un dato ed ottiene un certo valore $X$ è garantito che se rilegge non riceverà un dato che è stato scritto prima di $X$.
  \item \textbf{Read My Writes}: abbiamo coerenza forte rispetto alle modifiche effettuate dal client stesso mentre eventual per tutte le altre.
\end{itemize}
\end{frame}

\begin{frame}
\frametitle{Coerenze a confronto}
  \begin{center}
  \begin{tabular}{l*{3}{c}r}
\textbf{Guarantee}     & \textbf{Consistency} & \textbf{Performance} & \textbf{Availability} \\
\hline
\textbf{Strong Consistency}     & excellent & poor & poor   \\
\textbf{Eventual Consistency}   & poor & excellent & excellent \\
\textbf{Consistent Prefix}      & okay & good & excellent\\
\textbf{Bounded Staleness}      & good & okay & poor     \\
\textbf{Monotonic Reads}        & okay & good & good     \\
\textbf{Read My Writes}         & okay & okay & okay     \\
\end{tabular}
  \end{center}
\end{frame}


\begin{frame}[fragile]
\frametitle{Il baseball in pseudocodice}
\begin{lstlisting}
 Write ("visitors", 0);
 Write ("home", 0);
   for inning = 1 .. 9
     outs = 0;
     while outs < 3
       visiting player bats;
       for each run scored
         score = Read ("visitors");
         Write ("visitors", score + 1);
     outs = 0;
     while outs < 3
       home player bats;
       for each run scored
         score = Read ("home");
         Write ("home", score + 1);
 end game;
\end{lstlisting}
\end{frame}

\begin{frame}
\frametitle{Un esempio}  
\begin{center}
 \begin{tabular}{l*{10}{c}r}
  \textbf{Team}              & \textbf{1} & \textbf{2} & \textbf{3} & \textbf{4} & \textbf{5} & \textbf{6} & \textbf{7} & \textbf{8} & \textbf{9} & \textbf{RUNS} \\
 \hline
  \textbf{Visitors}     & 0 & 0 & 1 & 0 & 1 & 0 & 0 &  &  & 2 \\
  \textbf{Home}         & 1 & 0 & 1 & 1 & 0 & 2 &   &  &  & 5 \\
 \end{tabular}
\end{center}

\begin{center}
 \begin{tabular}{ |l|l| }
  \hline
  \multicolumn{2}{|c|}{Letture di punteggi possibili per ogni tipo di coerenza} \\
  \hline
  \textbf{Strong Consistency}   & 2-5 \\
  \hline
  \textbf{Eventual Consistency} & 0-0, 0-1, 0-2, 0-3, 0-4, 0-5, 1-0, 1-1, 1-2,\\
                                & 1-3, 1-4, 1-5, 2-0, 2-1, 2-2, 2-3, 2-4, 2-5 \\
  \hline
  \textbf{Consistent Prefix}    & 0-0, 0-1, 1-1, 1-2, 1-3, 2-3, 2-4, 2-5 \\
  \hline
  \textbf{Bounded Staleness}    & un inning indietro: 2-3, 2-4, 2-5 \\
  \hline  
  \textbf{Monotonic Reads}      & dopo aver letto 1-3: 1-3, 1-4, 1-5, 2-3, 2-4, 2-5 \\
  \hline  
  \textbf{Read My Writes}       & per chi scrive: 2-5 \\
                                & altri: eventual consistency \\
  \hline
 \end{tabular}
\end{center}
\end{frame}

\begin{frame}
 \frametitle{Soggetti del baseball}
 Nel gioco del baseball vi sono differenti soggetti con diversi ruoli, ognuno di essi necessita di un modello di coerenza diverso:\\
 \begin{itemize}
   \item Segnapunti
   \item Arbitro
   \item Radiocronista
   \item Giornalista sportivo
   \item Addetto alle statistiche
   \item Osservatore
 \end{itemize}
\end{frame}

\begin{frame}[fragile]
 \frametitle{Segnapunti}
  \begin{lstlisting}
    score = Read ("visitors");
    Write ("visitors", score + 1);
  \end{lstlisting}
  \pause
  \begin{block}{Coerenza desiderata}
    \begin{itemize}
      \item \alert{Strong consistency} è superflua, il segnapunti è l'unico ad effettuare delle write.
      \pause
      \item \alert{Read My Writes} è sufficiente.
    \end{itemize}     
  \end{block}
\end{frame}

\begin{frame}[fragile]
 \frametitle{Arbitro}
  \begin{lstlisting}
    if first half of 9th inning complete then
      vScore = Read ("visitors");
      hScore = Read ("home");
      if vScore < hScore
        end game;
  \end{lstlisting}
  \pause
 \begin{block}{Coerenza desiderata}
   L'arbitro necessita della \alert{Strong Consistency} per determinare la squadra vincitrice e concludere la partita.
 \end{block}
\end{frame}

\begin{frame}[fragile]
 \frametitle{Radiocronista}
  \begin{lstlisting}
    do {
        vScore = Read ("visitors");
        hScore = Read ("home");
        report vScore and hScore;
        sleep (30 minutes);
    }
  \end{lstlisting}
  \pause
  \begin{block}{Coerenza desiderata}
    \begin{itemize}
      \item \alert{Consistent Prefix} non è sufficiente, la lettura successiva può avvenire su un nodo non sincronizzato.\\
      Esempio: dopo aver letto lo score 2-5 è possibile leggere il punteggio 1-3.
      \pause
      \item \alert{Monotonic Reads} o \alert{Bounded Staleness} è necessaria. 
    \end{itemize}
  \end{block}
\end{frame}

\begin{frame}[fragile]
 \frametitle{Giornalista sportivo}
  \begin{lstlisting}
    While not end of game {
      drink beer;
      smoke cigar;
    }
    go out to dinner;
    vScore = Read ("visitors");
    hScore = Read ("home");
    write article;
  \end{lstlisting}
  \pause
  \begin{block}{Coerenza desiderata}
   \begin{itemize}
    \item \alert{Eventual Consistency} ?% è probabile che ritorni il risultato corretto dopo un ora dalla partita
    \pause
    \item \alert{Bounded Staleness} garantisce la sicurezza sul risultato finale.
   \end{itemize}  
  \end{block}
\end{frame}

\begin{frame}[fragile]
 \frametitle{Addetto alle statistiche}
  \begin{lstlisting}
    Wait for end of game;
    score = Read ("home");
    stat = Read ("season-runs");
    Write ("season-runs", stat + score);
  \end{lstlisting}
  \pause
  \begin{block}{Coerenza desiderata}
  Addetto alle statistiche necessita di due dati:
  \begin{itemize}
    \item Punteggio ultima partita (home):\\
    \begin{itemize}
	  \item \alert{Strong Consistency}: se effettuata il giorno stesso (lasso di tempo breve)
   	  \item \alert{Bounded Staleness}: se diversi giorni dopo
    \end{itemize}
    \pause
    \item Punteggi accumulati (season-runs):\\
    \begin{itemize}
  		\item \alert{Read My Writes}: se è l'unico ad effettuare le scritture    
		\item \alert{Strong Consistency}: altrimenti
    \end{itemize}

  \end{itemize}
   \end{block}
\end{frame}

\begin{frame}[fragile]
 \frametitle{Osservatore}
  \begin{lstlisting}
    do {
        stat = Read ("season-runs");
        discuss stats with friends;
        sleep (1 day);
    }
  \end{lstlisting}
  \pause
  \begin{block}{Coerenza desiderata}
    \alert{Eventual Consistency} è sufficiente, le statistiche vengono aggiornate ogni giorno
  \end{block}
\end{frame}